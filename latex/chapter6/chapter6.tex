
% CHAPTER 6

\chapter{CONCLUSION AND FUTURE WORKS}
\label{chp:conclusion}


\externaldocument{chapter1}
\externaldocument{chapter2}
\externaldocument{chapter3}
\externaldocument{chapter4}
\externaldocument{chapter5}

	
\section{Conclusion}
In this thesis work, we have proposed solutions to the dynamic formation control of heterogeneous mobile robots. These solutions are constructed with the help of two different subsystems, local positioning systems and formation control systems. 

A	local positioning system is designed to provide agents accurate position and velocity data. It is assumed that only a percentage of the swarm has position measurement sensors (e.g. a GNSS solution) due to the low sensor capabilities which is discussed in Section \ref{LOCAL POSITIONING SYSTEMS_ref}. So there is a necessity to implement a solution for estimating the positions and the velocities of the rest of the agents which do not have external position measurement sensors on their board.  A solution is proposed including the fusion of inertial measurements of the agents and the position data calculated with the help of the position beacons (i.e. the agents which have position measurement sensors on their boards). An algorithm that provides INS with the help of IMU sensors (i.e. Accelerometers, Gyroscopes, Magnetometers) is designed for each agent which propagates the translational accelerations of the agents in the Earth frame by implementing an AHRS system. Since this INS solution is vulnerable in position and velocity estimations due to the drift problems and measurement noise of the sensors, an external measurement of position is provided to each agent to correct their estimations based on local trilateration process. Local trilaterion process includes the distribution of the position data from the position beacons to the agents which do not have position sensors. Since it is assumed that the agents have limited communication capabilities, a solution to distribute the position data with minimum error to the agents farthest from the position beacons is provided with the help of route tables and assignment of rank values to the agents in the mesh network. Trilateration process is handled with the local neighbors of each agent to calculate the estimated position data. This position data is used in the Kalman observers as the external measurement data. Since it is not possible to execute the route table determination and the local trilateration process with the main process period of the agents (which controls the actuators, propagates INS measurements etc.) due to the computational complexity of the algorithm, a maximum period of time to execute localization process is determined with the help of an upper error bound allowable to achieve formation control with success. The maximum error norm for each agent is calculated below 3 meters with Monte Carlo simulations with the localization period of 3 seconds. This error value is chosen as the maximum tolerable value for the formation control system. So the period for the localization timer is chosen as 3 seconds.
       
In the second subsystem of the solution, we have introduced three types of methods based on potential field approach and shape partitioning approaches. Potential field approach implements virtual forces to the agents created by the desired formation shape, obstacles in the environment and the other agents. The X-Swarm theory is introduced to prove that the movement of an agent which is outside of the desired formation shape will be towards to the center of the swarm. The potential field based algorithm is designed to satisfy the conditions for the X-Swarm to direct the agents to the center of the swarm which is also moving towards to the center of the formation shape. It is observed that the agents are capable of homogeneously covering of the desired formation shape with the help of these artificial forces.

The other two methods based on shape partitioning approaches, are Bubble Packing method and randomized fractals method . These methods have different solutions to partition the desired formation shape into potential goal states. The algorithms which determines the assignment of the agents to these goal states in a decentralized manner is identical for these two methods. 

Bubble Packing algorithm introduces a method to partition the desired formation shape based on interbubble forces, similar to the Van der Waals forces between the molecular bonds, to distribute the bubbles homogeneously. It is widely used in mesh generation problems. The idea is to generate a mesh for a surface with identical bubbles to mimic a regular Voronoi diagram including the vertices representing the center of the bubbles. In our problem, the agents are represented by the  bubbles with equal radius' of their coverage circles described in Section \ref{Artificial_forces_ref}, and the algorithm is executed to generate the desired homogenous mesh to partition the shape into potential goal states. Each goal state has the information of agent type (i.e. coverage circle radius) to be placed at the final configuration. The Bubble Packing method have a similar approach with the potential fields since they both implement intermember$\&$interbubble forces to homogeneously distribute the agents$\&$bubbles in the desired formation$\&$surface 
       
Randomized fractals method is based on randomly distributing the coverage circles which represent the different agent types in the desired formation shape. This algorithm is faster than the Bubble Packing method since it randomly determines potential goal states. Because of this randomized approach, it is not applicable to implement a dynamic formation control solution with this method. 
       
The assignment process of the agents to the potential goal states is identical for these two different shape partitioning methods. First, the obstacles in the environment is augmented with Minkowski sums for different types of agents. The total coverage of the augmented obstacles represents the forbidden space for each agent. The free configuration space is calculated by extracting these forbidden spaces from the configuration space itself. Then each agent calculates the visibility graphs in the environment with augmented obstacles by including its current position and the potential goal states determined by the shape partitioning algorithms. Shortest paths in these visibility graphs to the each potential goal states are calculated with the help of Dijkstra's Algorithm. Hungarian algorithm is implemented to reach a global consensus on the assignment of the agents to the goal states based on minimizing the total displacement of the swarm by taken into consideration the distances of shortest paths to each goal states reported by each agent. 
       
Monte Carlo simulations with 1000 iterations is held with the same initial conditions and formation shapes for the three different proposed solution. To evaluate the performance of these methods, mesh quality, total energy consumption, settling time and dynamic formation performance metrics are calculated. As expected the worst mesh quality performance belongs to the randomized fractals method since it has an approach to randomly distribute the coverage circles in the desired formation shape. The potential field based method and the Bubble Packing method have similar mesh qualities because of their nature of implementing intermember$\&$interbubble forces which contributes the homogeneously distribution of the agents$\&$bubbles in the desired formation shape. On the other hand, potential field based method has the worst settling time and total displacement performance due to the lack of predetermined goal states for the agents in the desired formation shapes. The best performance is achieved with the Bubble Packing method in terms of these three metrics. Hence some different formation shape trials including special cases like a formation shape which includes obstacles are held with Bubble Packing method. 
       
At the end of the work, a hardware implementation of Bubble Packing method is held with 5 mobile robots from different sizes. Since this implementation is done at the  indoor environment and the number of agents is not sufficient, it is not possible to implement the local positioning system at this demonstration. The position and orientation data for the agents are calculated with image processing algorithms based on visual feedback provided by an E/O camera. A mesh network established with Zigbee modules and these modules are used to provide a communication backbone between the agents and the mission computer. The agents are designed as 3 axis omni-wheel mobile robots which have the capability to change position in the environment without the need for adapting their orientation. Each agent has its own processor module to execute the goal state decision process, control algorithms and step motor control routines. This feature demonstrates the decentralized manner of the solution in which  there is no central server deciding the final states of each agent in the solution. Several formation shape trials are done and the performance is evaluated with total energy consumption and settling time metrics. It is observed that the agents are capable to achieve different formation shapes as desired. This work is important to show the proposed solution about the formation control problem can be implemented in real time environment. The results represent a proof of concept (POC) of the thesis work rather than implementing the all details of the proposed solution.
       
\section{Future Works}
It is important to realize that there is a need to make hardware implementation with more agents within the environment. This kind of  implementation will give a more insight about the performance and the drawbacks about the system according to the tests held in simulation environment. So, the implementation with more number of agents in real time is one of the next steps for this project.
		
Even if the Bubble Packing method has the best performance in the sense of both mesh quality, total displacement and settling time issues, it is not a fully decentralized method since the shape partitioning process is executed on a central server node. Shape partitioning algorithm is not deterministic and it doesn't converge to the same potential goal states even if the initial states are identical. If it is possible to implement a method to partition the desired formation shape with a similar mesh quality performance in a more deterministic manner, it will be possible to distribute the partitioning process to individual agents. 
    
Since the main focus is not on the obstacle avoidance issue of the agents, all three methods are augmented with only potential fields created by the obstacles in the environment just to implement a basic obstacle avoidance. It is needed to implement a more complex obstacle avoidance algorithm (e.g. tangent bug algorithm) to avoid the cases in which some of the agents get into unwanted equilibriums under the effect of the desired setpoints and the obstacle forces.
    
In this thesis work, one of the main metrics to evaluate the performance of the proposed solutions is the total displacement of the swarm while achieving the desired formation shape. Moreover, in the shape partitioning methods the goal state assignment algorithms are designed to minimize the total displacement of the swarm. But, there may be cases in which some of the agents are more critical to reach the goal state or some of the agents are running out of battery$\&$energy. In such conditions, it will be appropriate to improve the goal state assignment algorithm to prioritize the agents to handle these kind of special cases. 
    
The main aim was to create a swarm with heterogeneous agents which can achieve a desired complex formation. One of the idea about this concept is to achieve some complex tasks collectively with the agents which have different individual capabilities. In our thesis work, we have focused on achieving complex formation shapes with heterogeneous mobile robots rather than achieving complex tasks after getting the desired formation shapes. The next step for this project is to add the capability of doing these kind of tasks with the agents from different capabilities.
		
		
		
		
		
		
		
		
		
		
		
		
		
		
		
		
		