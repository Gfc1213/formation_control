
% CHAPTER 6

\chapter{CONCLUSION AND FUTURE WORKS}
\label{chp:conclusion}


\externaldocument{chapter1}
\externaldocument{chapter2}
\externaldocument{chapter3}
\externaldocument{chapter4}
\externaldocument{chapter5}

	
\section{Conclusion}
In this thesis work, we proposed novel contributions to the dynamic formation control of heterogeneous mobile robots. These solutions are constructed with the help of two different subsystems, local positioning systems and formation control systems. 

A	local positioning system is designed to provide agents accurate position and velocity data. It is assumed that only a percentage of the swarm has position measurement sensors (e.g. a GNSS solution) due to the low sensor capabilities of the agents. So there is a necessity to implement a solution for estimating the positions and the velocities of the rest of the agents which do not have external position measurement sensors onboard.  We have proposed a solution including the fusion of inertial measurements of the agents and the position data calculated with the help of trilateration process. Propagation of the positions with the help of inertial measurements are vulnerable due to the drift problems and measurement noise of the sensors. Because of this reason, an external position measurement based on local trilateration process is provided to each agent to correct their estimations . Local trilateration process is handled with the help of position beacons which are direct neighbors of each agent. This position data is used in the Kalman observers as the external measurement data. It is not possible to execute the route table determination and the local trilateration process with the main process period of the agents (which controls the actuators, propagates inertial measurements etc.) due to the computational complexity of the algorithm. Because of this reason, we have defined a localization period to execute the trilateration process and call the update procedure of the Kalman observer. This period is chosen with the help of an upper error bound allowable to achieve formation control with success.
       
In the second subsystem of the solution, we have introduced three types of methods based on potential field approach and shape partitioning approaches. Potential field approach implements artificial forces to the agents created by the desired formation shape, obstacles in the environment and the other agents. We have defined the individual control laws for agents as sum of these different artificial forces. We have observed that the agents are capable of homogeneously covering the desired formation shape with the help of these artificial forces.

The other two methods based on shape partitioning approaches, are Bubble Packing method and Randomized Fractals method . These methods have different solutions to partition the desired formation shape into potential goal states. The algorithms which determines the assignment of the agents to these goal states in a decentralized manner is identical for these two methods. 

Bubble Packing algorithm introduces a method to partition the desired formation shape based on interbubble forces, similar to the Van der Waals forces between the molecular bonds, to distribute the bubbles homogeneously. It is widely used in mesh generation problems. The idea is to generate a mesh for a surface with identical bubbles to mimic a regular Voronoi diagram including the vertices representing the center of the bubbles. In our problem, the agents are represented by bubbles with equal radius' of their coverage circles and the algorithm is executed to generate the desired homogeneous mesh to partition the shape into potential goal states. Each goal state has the information of agent type (i.e. coverage circle radius) to be placed at the final configuration. The Bubble Packing method have a similar approach with the potential fields since they both implement intermember$\&$interbubble forces to homogeneously distribute the agents$\&$bubbles in the desired formation$\&$surface.
       
Randomized fractals method is based on randomly distributing the coverage circles which represent the different agent types in the desired formation shape. This algorithm is faster than the Bubble Packing method since it randomly determines potential goal states. Because of this randomized approach, it is not applicable to implement a dynamic formation control solution with this method. 
       
The assignment process of the agents to the potential goal states is identical for these two different shape partitioning methods. First, the obstacles in the environment is augmented with Minkowski sums for different types of agents. The total coverage of the augmented obstacles represents the forbidden space for each agent. The free configuration space is calculated by extracting these forbidden spaces from the configuration space itself. Then each agent calculates their own visibility graphs in the environment with augmented obstacles by including its current position and the potential goal states determined by the shape partitioning algorithms. Shortest paths in these visibility graphs to the each potential goal states are calculated with the help of Dijkstra's Algorithm. Hungarian algorithm is implemented to reach a global consensus on the assignment of the agents to the goal states based on minimizing the total displacement of the swarm by taken into consideration the costs reported by each agent. 
       
Monte Carlo simulations with 1000 iterations is run with the same initial conditions and formation shapes for the three different proposed solution. To evaluate the performance of these methods, mesh quality, total displacement, settling time and dynamic formation performance metrics are calculated. As expected the worst mesh quality performance belongs to the Randomized Fractals method since it has an approach to randomly distribute the coverage circles in the desired formation shape. The Artificial Forces method and the Bubble Packing method have similar mesh qualities because of their nature of implementing intermember$\&$interbubble forces which contributes the homogeneously distribution of the agents$\&$bubbles in the desired formation shape. On the other hand, Artificial Forces method has the worst settling time and total displacement performance due to the lack of predetermined goal states for the agents in the desired formation shapes. The best performance is achieved with the Bubble Packing method in terms of these three metrics.
       
At the end of the work, a hardware implementation of Bubble Packing method is done with 5 mobile robots from different sizes. Since this implementation is done at the  indoor environment and the number of agents is not sufficient, it is not possible to implement the local positioning system at this demonstration. The position and heading data for the agents are calculated with image processing algorithms based on visual feedbacks provided by an E/O camera. A mesh network established with Zigbee modules and these modules are used to provide a communication backbone between the agents and the mission computer. The agents are designed as 3 axis omni-wheel mobile robots which have the capability to change position in the environment without the need for adapting their heading. Each agent has its own processor module to execute the goal state decision process, artificial force calculations, control algorithms and step motor control routines. This feature demonstrates the decentralized manner of the solution. Several formation shape trials are done and the performance is evaluated with total displacement and settling time metrics. It is observed that the agents are capable to achieve different formation shapes as desired. This work is important to show that the proposed solution about the formation control problem can be implemented in real time environment. The results represent a proof of concept (POC) of the thesis work rather than implementing th extensive detailed evaluations of the proposed solution that we preferred to do within the Gazebo simulation environment.

In this project, we aim to implement a complete solution to the formation shape generation problem including both local positioning system and formation control system. With the usage of local positioning system, it is possible to implement our solution for different workspaces in both indoor and outdoor environments. Moreover, we have implemented two novel methods on formation control problem; Bubble Packing and Randomized Fractals methods. The usage of these shape partitioning based algorithms eliminates some drawbacks related with potential field based approaches like unwanted equilibrium states and higher settling times. 
       
\section{Future Works}
It is important to realize that there is a need to make hardware implementation with more agents within the environment. This kind of  implementation will give a more insight about the performance and the drawbacks of the system with the tests held in real time environment. The implementation with more number of agents in real time is one of the next steps for this project.
		
Even if the Bubble Packing method has the best performance in the sense of both mesh quality, total displacement and settling time issues, it is not a fully decentralized method since the shape partitioning process is executed on a central server node. Shape partitioning algorithm is not deterministic and it doesn't converge to the same potential goal states even if the initial states are identical. If it is possible to implement a method to partition the desired formation shape with a similar mesh quality performance in a more deterministic manner, it will be possible to distribute the partitioning process to individual agents. 
    
Since our main focus is not on the obstacle avoidance of the agents, all three methods are augmented with only potential fields created by the obstacles in the environment just to implement a basic obstacle avoidance. It is needed to implement a more convenient obstacle avoidance algorithm (e.g. tangent bug algorithm) to avoid the cases in which some of the agents get into unwanted equilibriums under the effect of the desired setpoints and the obstacle forces.
    
In this thesis work, one of the main metrics to evaluate the performance of the proposed solutions is the total displacement of the swarm while achieving the desired formation shape. Moreover, in the shape partitioning methods the goal state assignment algorithms are designed to minimize the total displacement of the swarm. But, there may be cases in which some of the agents are more critical to reach the goal state or some of the agents are running out of battery$\&$energy. In such conditions, it will be appropriate to improve the goal state assignment algorithm to prioritize the agents to handle these kind of special cases. 
    
Our main aim is to create a swarm with heterogeneous agents to achieve a desired complex formation. One of the purposes related with this concept is to achieve some complex tasks collectively with the agents which have different individual capabilities. In our thesis work, we have focused on achieving complex formation shapes with heterogeneous mobile robots rather than achieving complex tasks after getting the desired formation shapes. The next step for this project is to add the capability of doing these kind of tasks with the agents from different capabilities.

We have implemented our formation control algorithms in two dimensional workspaces. It is possible to augment our solutions to three dimensional workspaces. One of the future works for this project is to adapt our local positioning and formation control algorithms for three dimensional environments. With this work it will be possible to test the related algorithms in real world applications such as search$\&$rescue operations, surveillance and coverage operations.
		
		
		
		
		
		
		
		
		
		
		
		
		
		
		
		